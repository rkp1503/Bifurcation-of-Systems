\section{Analyzing the system for a default set of parameters}\label{sec:analyzing-the-system-for-a-default-set-of-parameters}
Let's consider the system when $\alpha=-1,\beta=1,\gamma=0$:
\begin{equation}
    \begin{dcases}
        \begin{aligned}
            \diff[]{x}{t} &= f(x,y;-1,1,0) = \left(x^2-1-y\right)\left((x-1)+(x-1)^3-(y-1)\right)\left(x-2y\right)\\
            \diff[]{y}{t} &= g(x,y;-1,1) = x-y
        \end{aligned}
    \end{dcases}
    \label{eq:4}
\end{equation}

\subsection{Equations to solve to find all fixed points}\label{subsec:equations-to-solve-to-find-all-fixed-points}
In order to find all the fixed points, we will need to solve the following equations:
\begin{equation*}
    \begin{dcases}
        \begin{aligned}
            x &= x^2-1\\
            x &= (x-1)+(x-1)^3+1\\
            x &= \frac{x}{2}
        \end{aligned}
    \end{dcases}
\end{equation*}

\subsection{Analyzing all fixed points via the Jacobian Matrix}\label{subsec:analyzing-all-fixed-points-via-the-jacobian-matrix}
Using the provided software and the prior analysis, the fixed points for this system are:
\begin{equation*}
    \left(x^*,y^*\right) = \left\{\left(\frac{1-\sqrt{5}}{2}, \frac{1-\sqrt{5}}{2}\right), \left(0,0\right), \left(1,1\right), \left(\frac{1+\sqrt{5}}{2}, \frac{1+\sqrt{5}}{2}\right)\right\}
\end{equation*}
For this system, the Jacobian matrix is:
\begin{equation*}
    \textbf{J} = \begin{bmatrix}
        \pdiff[]{f}{x} & \pdiff[]{f}{y}\\
        1 & -1
    \end{bmatrix}
\end{equation*}
where
\begin{align*}
    \pdiff[]{f}{x} &= 6x^5-10x^4y-15x^4+20x^3y+12x^3+6x^2y^2-12x^2y+6x^2-8xy^2-16xy-8x+9y^2+10y+1\\
    \pdiff[]{f}{y} &= -2x^5+5x^4+4x^3y-4x^3-8x^2y-8x^2+18xy+10x-6y^2-8y-2
\end{align*}
Lets give these fixed points names:
\begin{equation*}
    P_1=\left(\frac{1-\sqrt{5}}{2}, \frac{1-\sqrt{5}}{2}\right),\quad P_2=\left(0,0\right),\quad P_3=\left(1,1\right),\quad P_4=\left(\frac{1+\sqrt{5}}{2}, \frac{1+\sqrt{5}}{2}\right)
\end{equation*}
For the four fixed points, the Jacobian Matrix, their traces and determinants are:
\begin{alignat*}{3}
    P_1:&\quad \textbf{J} = \begin{bmatrix}
        1+\sqrt{5} & \frac{3+\sqrt{5}}{2}\\
        1 & -1
    \end{bmatrix} &&\implies \begin{dcases}
    \begin{aligned}
        \tau &= \sqrt{5}>0\\
        \Delta &= -\frac{5+3\sqrt{5}}{2}<0
    \end{aligned}
    \end{dcases}\\
    P_2:&\quad \textbf{J} = \begin{bmatrix}
        1 & -2\\
        1 & -1
    \end{bmatrix} &&\implies \begin{dcases}
    \begin{aligned}
        \tau &= 0\\
        \Delta &= 1>0
    \end{aligned}
    \end{dcases}\\
    P_3:&\quad \textbf{J} = \begin{bmatrix}
        1 & -1\\
        1 & -1
    \end{bmatrix} &&\implies \begin{dcases}
    \begin{aligned}
        \tau &= 0\\
        \Delta &= 0
    \end{aligned}
    \end{dcases}\\
    P_4:&\quad \textbf{J} = \begin{bmatrix}
        1-\sqrt{5} & \frac{3-\sqrt{5}}{2}\\
        1 & -1
    \end{bmatrix} &&\implies \begin{dcases}
    \begin{aligned}
        \tau &= -\sqrt{5}<0\\
        \Delta &= \frac{-5+3\sqrt{5}}{2}>0
    \end{aligned}
    \end{dcases}
\end{alignat*}
Here we can classify some of these fixed points. For $P_1$, since $\Delta<0$, this implies that $P_1$ is a Saddle Node. Also, since $\tau>0$, this implies that $P_1$ is Unstable. Therefore $P_1$ is an Unstable Saddle Node. For $P_2$, since $\Delta>0$ and $\tau^2-4\Delta<0$, this implies that we have a Center. Furthermore, since $\tau=0$, this implies that $P_2$ is Neutrally Stable. Therefore, $P_2$ is a Neutrally Stable Center. For $P_3$, we have $\tau=\Delta=0$. Since this system is a nonlinear system, we cannot draw any conclusions about the type of fixed point $P_3$ is based on this analysis alone. We will need to do more work. Using the software provided, we will look into trajectories close to $P_3$. Based on \figref{fig:4}, we can say that $P_3$ is a Saddle-Node. For $P_4$, since $\Delta>0$ and $\tau^2-4\Delta>0$, this implies that $P_4$ is a Node. Furthermore, since $\tau<0$, this implies that $P_4$ is Stable. Therefore, $P_4$ is a Stable Node.

\subsection{Computing Eigenvalues and Eigenvectors for all fixed points}\label{subsec:computing-eigenvalues-and-eigenvectors-for-all-fixed-points}
Now we will find the eigenvalues and their corresponding eigenvectors for all fixed points present in the system. Starting with $P_1$ the eigenvalues are:
\begin{align*}
    p(\lambda) &= \det\left(\textbf{J}-\lambda I\right) = \begin{vmatrix}
        1+\sqrt{5}-\lambda & \frac{3+\sqrt{5}}{2}\\
        1 & -1-\lambda
    \end{vmatrix}\\
    0 &= (1+\sqrt{5}-\lambda)(-1-\lambda)-1\left(\frac{3+\sqrt{5}}{2}\right)\\
    0 &= \lambda^2-\sqrt{5}\lambda - \frac{5+3\sqrt{5}}{2}\\
    \implies \lambda &= \frac{-\left(-\sqrt{5}\right) \pm \sqrt{\left(-\sqrt{5}\right)^2-4(1)\left(-\frac{5+3\sqrt{5}}{2}\right)}}{2(1)}\\
    \Aboxed{\lambda &= \left\{\frac{\sqrt{5} - \sqrt{15+6\sqrt{5}}}{2}, \frac{\sqrt{5} + \sqrt{15+6\sqrt{5}}}{2}\right\} = \{-1.5473, 3.7834\}}
\end{align*}
and their respective eigenvectors are:
\begin{align*}
    \vec{v} &= \left\{\begin{bmatrix}
        \frac{3+\sqrt{5}}{2}\\
        -\left(1+\sqrt{5}-\frac{\sqrt{5} - \sqrt{15+6\sqrt{5}}}{2}\right)
    \end{bmatrix}, \begin{bmatrix}
        \frac{3+\sqrt{5}}{2}\\
        -\left(1+\sqrt{5}-\frac{\sqrt{5} + \sqrt{15+6\sqrt{5}}}{2}\right)
    \end{bmatrix}\right\}\\
    \Aboxed{\vec{v} &= \left\{\begin{bmatrix}
        \frac{3+\sqrt{5}}{2}\\
        \frac{\sqrt{5} - \sqrt{15+6\sqrt{5}}}{2}-1-\sqrt{5}
    \end{bmatrix}, \begin{bmatrix}
        \frac{3+\sqrt{5}}{2}\\
        \frac{\sqrt{5} + \sqrt{15+6\sqrt{5}}}{2}-1-\sqrt{5}
    \end{bmatrix}\right\} = \left\{\begin{bmatrix}
        2.6180\\
        -4.7834
    \end{bmatrix}, \begin{bmatrix}
        2.6180\\
        0.5473
    \end{bmatrix}\right\}}
\end{align*}
For $P_2$, the eigenvalues are:
\begin{align*}
    p(\lambda) &= \det\left(\textbf{J}-\lambda I\right) = \begin{vmatrix}
        1-\lambda & -2\\
        1 & -1-\lambda
    \end{vmatrix}\\
    0 &= (1-\lambda)(-1-\lambda)-1(-2)\\
    0 &= \lambda^2+1\\
    \implies \Aboxed{\lambda &= \left\{-i, i\right\}}
\end{align*}
and their respective eigenvectors are:
\begin{align*}
    \vec{v} &= \left\{\begin{bmatrix}
        -2\\
        -\left(1+i\right)
    \end{bmatrix}, \begin{bmatrix}
        -2\\
        -\left(1-i\right)
    \end{bmatrix}\right\}\\
    \Aboxed{\vec{v} &= \left\{\begin{bmatrix}
        -2\\
        -1-i
    \end{bmatrix}, \begin{bmatrix}
        -2\\
        i-1
    \end{bmatrix}\right\}}
\end{align*}
Since we have complex eigenvalue and complex eigenvectors, this implies that the fixed point is either a spiral or a center, which has a sense of rotation. The flow for this fixed point rotates clockwise.\\
For $P_3$, the eigenvalues are:
\begin{align*}
    p(\lambda) &= \det\left(\textbf{J}-\lambda I\right) = \begin{vmatrix}
        1-\lambda & -1\\
        1 & -1-\lambda
    \end{vmatrix}\\
    0 &= (1-\lambda)(-1-\lambda)-1(-1)\\
    0 &= \lambda^2\\
    \implies \Aboxed{\lambda &= \left\{0\right\}}
\end{align*}
and their respective eigenvectors are:
\begin{align*}
    \vec{v} &= \left\{\begin{bmatrix}
        -1\\
        -\left(1\right)
    \end{bmatrix}\right\}\\
    \Aboxed{\vec{v} &= \left\{\begin{bmatrix}
        -1\\
        -1
    \end{bmatrix}\right\}}
\end{align*}
For $P_4$, the eigenvalues are:
\begin{align*}
    p(\lambda) &= \det\left(\textbf{J}-\lambda I\right) = \begin{vmatrix}
        1-\sqrt{5}-\lambda & \frac{3-\sqrt{5}}{2}\\
        1 & -1-\lambda
    \end{vmatrix}\\
    0 &= (1-\sqrt{5}-\lambda)(-1-\lambda)-1\left(\frac{3-\sqrt{5}}{2}\right)\\
    0 &= \lambda^2+\sqrt{5}\lambda+\frac{-5+3\sqrt{5}}{2}\\
    \implies \lambda &= \frac{-\left(\sqrt{5}\right) \pm \sqrt{\left(\sqrt{5}\right)^2-4\left(\sqrt{5}\right)\left(\frac{-5+3\sqrt{5}}{2}\right)}}{2\left(1\right)}\\
    \Aboxed{\lambda &= \left\{\frac{-\sqrt{5}-\sqrt{15-6\sqrt{5}}}{2}, \frac{-\sqrt{5}+\sqrt{15-6\sqrt{5}}}{2}\right\} = \left\{-1.7472, -0.4888\right\}}
\end{align*}
and their respective eigenvectors are:
\begin{align*}
    \vec{v} &= \left\{\begin{bmatrix}
        \frac{3-\sqrt{5}}{2}\\
        -\left(1-\sqrt{5}-\frac{-\sqrt{5}-\sqrt{15-6\sqrt{5}}}{2}\right)
    \end{bmatrix}, \begin{bmatrix}
        \frac{3-\sqrt{5}}{2}\\
        -\left(1-\sqrt{5}-\frac{-\sqrt{5}+\sqrt{15-6\sqrt{5}}}{2}\right)
    \end{bmatrix}\right\}\\
    \Aboxed{\vec{v} &= \left\{\begin{bmatrix}
        \frac{3-\sqrt{5}}{2}\\
        \sqrt{5}-1-\frac{\sqrt{5}+\sqrt{15-6\sqrt{5}}}{2}
    \end{bmatrix}, \begin{bmatrix}
        \frac{3-\sqrt{5}}{2}\\
        \sqrt{5}-1+\frac{-\sqrt{5}+\sqrt{15-6\sqrt{5}}}{2}
    \end{bmatrix}\right\} = \left\{\begin{bmatrix}
        0.3820\\
        -0.5112
    \end{bmatrix}, \begin{bmatrix}
        0.3820\\
        0.7472
    \end{bmatrix}\right\}}
\end{align*}

\subsection{Graphical representation of the system}\label{subsec:graphical-representation-of-the-system}
\figref{fig:5} shows the nullclines for the system with these sets of parameters. \figref{fig:6} shows the phase portrait for the system with these sets of parameters. It contains all the fixed points, trajectories to show the overall flow, arrows to indicate the direction of the flow with increasing time, eigenvector directions, and manifolds.

\subsection{Graphical analysis of the system}\label{subsec:graphical-analysis-of-the-system}
We have one fixed point that has complex eigenvalues, which is $(0,0)$. Referring to \figref{fig:8}, we can see that sense of rotation near this fixed point is clockwise, which is what we predicted in Section 4.3. We have one fixed point that is a Stable Node, which is the fixed point $\left(\frac{1+\sqrt{5}}{2}, \frac{1+\sqrt{5}}{2}\right)$. For this Stable Node, there are trajectories that go along one of the eigenvector directions.
